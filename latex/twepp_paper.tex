\documentclass[a4paper,11pt]{article}
\pdfoutput=1 % if your are submitting a pdflatex (i.e. if you have
             % images in pdf, png or jpg format)

\usepackage{jinstpub} % for details on the use of the package, please
                     % see the JINST-author-manual

\usepackage{lineno}
\linenumbers

\title{Development of a high bandwidth readout chain for the CMS Phase-2 pixel upgrade}

%% %simple case: multiple authors, same institution
\author{C. Smith}
\affiliation{The University of Kansas,\\Lawrence, Kansas 66045, USA}

% \affiliation{The University of Kansas\\1251 Wescoe Hall Dr.\\Lawrence, KS 66045, United States}

% From CMS publication:
% The University of Kansas, Lawrence, Kansas 66045, USA
% From KU physics website:
% The University of Kansas, 1251 Wescoe Hall Dr. Lawrence, KS 66045, United States

% more complex case: 4 authors, 3 institutions, 2 footnotes
% \author[a,b,1]{F. Irst,\note{Corresponding author.}}
% \author[c]{S. Econd,}
% \author[a,2]{T. Hird\note{Also at Some University.}}
% \author[c,2]{and Fourth}

% The "\note" macro will give a warning: "Ignoring empty anchor..."
% you can safely ignore it.

% \affiliation[a]{One University,\\some-street, Country}
% \affiliation[b]{Another University,\\different-address, Country}
% \affiliation[c]{A School for Advanced Studies,\\some-location, Country}

% e-mail addresses: only for the corresponding author
\emailAdd{caleb.smith@ku.edu}

\abstract{
The CMS collaboration is building a new inner tracking pixel detector for the High-Luminosity LHC. Each pixel chip will be controlled with a single serial input stream at 160 Mbps and will send out data via four CML 1.28 Gbps outputs. The modules will be connected with up to 1.6 m long low-mass electrical links to the low power gigabit transceivers (lpGBT) and versatile transceivers (VTRx+) that send the data optically to off-detector electronics at 10 Gbps. The development and the characterization of these components is presented along with system tests of the readout chain.
}

\keywords{Only keywords from JINST's keywords list please}

% \arxivnumber{1234.56789} % only if you have one

% \collaboration{\includegraphics[height=17mm]{example-image}\\[6pt]
%   XXX collaboration}
% or
\collaboration[c]{on behalf of the CMS collaboration}


% if you write for a special issue this may be useful
\proceeding{Topical Workshop on Electronics for Particle Physics\\
  September 20--24, 2021\\
  Online event}

\begin{document}
\maketitle
\flushbottom

% Comment on abbreviations
% We suggest not to abbreviate: ``section'', ``appendix'', ``figure''
% and ``table'', but ``eq.'' and ``ref.'' are welcome. Also, please do
% not use \texttt{\textbackslash emph} or \texttt{\textbackslash it} for
% latin abbreviaitons: i.e., et al., e.g., vs., etc.

\section{First Section}
\label{sec:first}

Here lies the first section. Put content here.

\section{Second Section}
\label{sec:second}

Here lies the second section. Put content here.

\paragraph{Paragraphs example}
Here is an example paragraph. Place text here.

\section{Third Section}
\label{sec:third}

Here lies the third section. Put content here.

% Appendices

\appendix
\section{Example appendix title}

Here is this appendix.

\acknowledgments

Put acknowledgments here.

\paragraph{Note added.}
Notes if needed.

% We suggest to always provide author, title and journal data:
% in short all the informations that clearly identify a document.

\begin{thebibliography}{99}

\bibitem{a}
Author, \emph{Example Title},
arxiv:1234.5678.

% \bibitem{a}
% Author, \emph{Title}, \emph{J. Abbrev.} {\bf vol} (year) pg.
%
% \bibitem{b}
% Author, \emph{Title},
% arxiv:1234.5678.
%
% \bibitem{c}
% Author, \emph{Title},
% Publisher (year).

% Please avoid comments such as "For a review'', "For some examples",
% "and references therein" or move them in the text. In general,
% please leave only references in the bibliography and move all
% accessory text in footnotes.

% Also, please have only one work for each \bibitem.

\end{thebibliography}
\end{document}
